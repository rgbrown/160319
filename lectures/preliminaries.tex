\part{Preliminaries}
\chapter{Introduction}
In this course, you are going to learn a lot about mathematical modelling.
The first half of the course is concerned with filling in some of the
important gaps in your mathematical toolkit.  

We are going to largely restrict ourselves to modelling using Ordinary
Differential Equations (ODEs). We can learn how maths can be applied
effectively in a great many different scenarios using ODEs, and plus you
already have some prior exposure to these, so it's not all new! 

Now, probably your experience of differential equations has been largely
constrained to finding analytical solutions to them - given a problem,
you've searched for an appropriate method to solve it, perhaps separation
of variables or something more exotic, and then used all of the provided
information about your problem to come up with a solution. This is but one
of three approaches that we will use for differential equations, and the
one that we will use the least! The other two are 
\begin{itemize}
    \item Graphical methods
    \item Numerical methods
\end{itemize}
The emphasis in this course is on the qualitative behaviour of dynamical
systems, which for us will mean systems governed by differential or
difference equations. We are not so much concerned about specific
solutions, but rather how the system behaves for different choices of
parameters or initial conditions.

We will learn how to use \emph{phase portraits} and \emph{bifurcation
diagrams}, special pictures that illustrate the dynamics of systems to
interpret their behaviour, and we'll usually be using numerical methods to
simulate our differential equations, as the types of equations encountered
during modelling are rarely solvable analytically.

\section{General systems}
We are going to be concerned with systems of differential equations of the
form 
\begin{equation}
    \begin{split}
        \dot{x_1} &= f_1(x_1, \ldots, x_n) \\
        &\vdots \\
        \dot{x_n} &= f_n(x_1, \ldots, x_n)
    \end{split}
\end{equation}
The overdots signify differentiation with respect to time $t$. These
systems are known as systems of \emph{autonomous} differential equations,
as there is no $t$ dependence in their right hand sides. Higher order
differential equations, such as 
\begin{equation*}
    \ddot{x} + a\dot{x} + x = 0
\end{equation*}
can readily be put into this format by the standard trick of defining the
variables $x_1 = x$, $x_2 = \dot{x}$. We then have the system
\begin{equation*}
    \begin{split}
        \dot{x_1} &= x_2 \\
        \dot{x_2} &= -x_1 - ax_2.
    \end{split}
\end{equation*}
We just define our $x_i$ to be the $(i-1)$-th derivative of $x$,
stopping one before we get to the highest order derivative present.

\section{One-dimensional systems}
Right. Let's start with the simplest case, just one differential equation
in one variable,
\begin{equation*}
    \dot{x} = f(x).
\end{equation*}
In fact, let's start with a specific example,
\begin{equation*}
    \dot{x} = \sin x.
\end{equation*}
We are going to try to analyse this by developing some graphical methods,
rather than relying on being able to solve it. Which, by the way, we could
if we wanted to -- this happens to be a differential equation that
\emph{does} have a closed-form solution. OK, let's do that real quick, just
to remember how painful and tedious an exercise this is. We start by
separating variables
\begin{equation*}
    \frac{1}{\sin x}\frac{dx}{dt} = 1
\end{equation*}
which gives
\begin{equation*}
    t = \int \csc x \, dx = \ln \abs{\csc x + \cot x} + C.
\end{equation*}
Putting in an initial condition $x(0) = x_0$ gives
\begin{equation*}
    t = \ln \abs{\frac{\csc{x_0} + \cot x_0}{\csc x + \cot x}}
\end{equation*}
Say something about that without getting a headache, I dare you. Can you
tell me
\begin{enumerate}
    \item With $x_0 = 1$, describe what happens to the solution as $t \to
        \infty$.
    \item What happens to $x$ as $t \to \infty$ for an arbitrary initial
        condition $x_0$?
\end{enumerate}
This kind of qualitative information is very difficult to glean from this
closed-form solution. We need a better way.

\section{A geometrical approach}
Let's now imagine that our equation $\dot{x} = \sin x$ represents the
velocity of a particle on the $x$ axis. When our particle is at position
$x_0$, it has velocity $f(x_0)$, and if that's positive, it means it's
going to zoom off in the positive $x$ direction, or, if it's negative, it
will head left.

        

